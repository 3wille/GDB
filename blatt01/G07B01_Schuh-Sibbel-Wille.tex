\documentclass[ngerman]{gdb-aufgabenblatt}

\usepackage[latin9]{inputenc}
\renewcommand{\Aufgabenblatt}{1}
\renewcommand{\Ausgabedatum}{Mi. 15.10.2014}
\renewcommand{\Abgabedatum}{Fr. 31.10.2014}
\renewcommand{\Gruppe}{Schuh, Sibbel, Wille}
\usepackage[T1]{fontenc}

\begin{document}


\section{Informationssysteme}
\begin{itemize}
	\item[a)] \textbf{Charakterisierung:}\\
		Laut Skript:  Ein Informationssystem (IS) besteht aus Menschen und Maschinen, die Informationen erzeugen und/oder benutzen und die durch Kommunikationsbeziehungen miteinander verbunden sind. \\
		\\
Aufgaben eines rechnergest�tzten Informationssystemes:
		\begin{itemize}
		\item Erfassung von Daten
		\item Speicherung von Daten
		\item Bearbeitung von Daten
		\end{itemize}
\item[b)] \textbf{logische Datenunabh�ngigkeit} beschreibt die Kapselung der Daten von der Anwendung. Das heisst, wenn die Struktur der Daten ver�ndert wird, funktioniert das Programm weiterhin und umgekehrt.\\
\textbf{logische Datenunabh�ngigkeit} besagt, dass die logischen Daten, wie sie der Benutzer sieht, von den physischen Daten auf der Festplatte getrennt sind. Somit k�nnen Daten auf der Festplatte reorganisiert werden, ohne das der Benutzer davon etwas merkt.
\item[c)] \textbf{Beispiele:}\\
\begin{itemize}
\item \textbf{Krankenhaus:} \\
Im Krankenhaus werden t�glich sehr viele Daten vorallem von Patienten erfasst, bearbeitet und gespeichert. Es werden Daten aus anderen Systemen, wie zum Beispiel der Krankenkasse bezogen und dort hin versand. Die Konsistenz dieser Daten ist von besonderer Bedeutung, da im Zweifelsfall Leben davon abh�ngen. Typische Vorg�nge sind  das Aufnehmen von neuen Partienten, Zuordnung von �rzten und Zimmern und Vergabe von Terminen bei verschiedenen Abteilungen. 
\item \textbf{B�rse:}
An der B�rse m�ssen im Millisekundentakt Transaktionen durchgef�hrt werden, die alle eindeutig zu einem Zeitpunkt zugeordnet werden m�ssen, da sich die Kurse blitzschnell �ndern k�nnen. Es werden Aktien gekauft und verkauft, dabei muss immer der aktuelle Kurs beachtet werden 
\item \textbf{Verbrecherdatenbank:}
Schneller Zugriff auf fr�here F�lle erm�glicht Mustererkennung und damit m�glicherweise schnellere Aufkl�rung von Verbrechen. Erm�glicht Identifikation durch biometrische Erkennung. Katalogisierung von Beweismitteln.
\end{itemize}
\end{itemize}

\section{Miniwelt}
\begin{itemize}
\item[a)] \textbf{Elemente} : 
\begin{itemize}
\item \textbf{Mitspieler}: Name, E-Mail Adresse
\item \textbf{Gemeinschaft}: Gr�nder, Mitspieler, Wettbewerbe
\item \textbf{Wettbewerbe}: Begegnungen,
\item \textbf{Begegnungen}: Ergebnisse, Tipps
\item \textbf{Tipp}: Mitspieler, Wette
\end{itemize} 
\textbf{Vorg�nge:}
\begin{itemize}
\item neuen Spieler Anmelden 
\item Gemeinschaft erstellen 
\item Mitglied zur Gemeinschaft hinzuf�gen/entfernen 
\item Begegnungen hinzuf�gen
\item Ergebnisse eintragen
\item Tipp abgeben
\item Punktestand abfragen
\item Ergebnisse abfragen
\item Punkte berechnen
\end{itemize}

\item[b)]
\begin{itemize}
\item[] \textbf{Kontrolle �ber die operationalen Daten:}\\
In der Tippspiel Anwendung sollen die Spieler auf Begegnungen tippen k�nnen. Danach sollen alle bisherigen Ergebnisse und die Punktzahlen angezeigt werden. Zu Beginn muss der Gr�nder einer Tippgemeinschafft die Mitspieler einladen, die Begegnungen erstellen und sp�ter die Ergebnisse dieser eintragen k�nnen.

\item[] \textbf{Leichte Handhabbarkeit der Daten:}\\
Durch eine Sinnvolle Strukturierung der Daten, k�nnen simplere Abrufe erm�glicht werden. Weiterhin sollen Daten nur einmal gespeichert werden. So kann z.B. die Gesamtpunktzahl eines Spielers aus den Punkten der Begegnungen berechnet werden,
\item[] \textbf{Kontrolle der Datenintegrit�t:}\\
Es m�ssen Zugriffsrechte beachtet werden, da nur der Gr�nder der Gemeinschaft administrieren k�nnen sollte. So darf als Beispiel nur er Ergebnisse eintragen. 
\item[] \textbf{Leistung und Skalierbarkeit:}\\
Das System muss so entworfen sein, dass hohe Zugriffszahlen keine hohen Zugriffszeiten erzeugen und erst recht nicht zu Verlust von Daten f�hren.
\item[] \textbf{Hoher Grad an Daten-Unabh�ngigkeit:}\\
Durch kleine �nderungen an der Datenbank oder der Anwendung soll keine Inkonsistenz entstehen und das jeweils andere unver�ndert weiterbenutzbar bleiben.
\end{itemize}
\end{itemize}

\section{Transaktionen}
Bei \textbf{Zeitpunkt A} wird von dem Konto mit der ID=5 1000 Geldeinheiten abgezogen, jedoch nicht auf das Empf�ngerkonto eingezahlt. Somit sind die Daten inkonsistent. \\
Bei \textbf{Zeitpunkt B} bleiben die Daten konsistent, das Geld wird sauber �bertragen, jedoch wird nach der Transaktion nur das Empf�ngerkonto und nicht das Senderkonto angezeigt.\\



\section{Warm-Up MySQL}
\begin{itemize}
\item[a)] \texttt{CREATE TABLE gdb\_gruppe052.user (...);} erstellt eine neue Tabelle \texttt{user} in der Datenbank \texttt{gdb\_gruppe052} mit den Spalten \texttt{id}, \texttt{name} und \texttt{passwort}. 

Mit dem Befehl \texttt{INSERT INTO gdb\_gruppe052.user (...) VALUES (...);} wird in die eben erstellte Tabelle ein Benutzer mit \texttt{id=1}, \texttt{name=``gdbNutzer''} und \texttt{passwort=``geheim''} eingef�gt.

\item[b)]
Mit \texttt{SELECT * FROM gdb\_gruppe052.user WHERE name = '"gdbNutzer"';} wird der in a) hinzugef�gte Benutzer abgefragt.

Mit \texttt{DROP TABLE gdb\_gruppe052.user;} wird die in a) erstellte Tabelle wieder gel�scht.

\item[c)]
Die Drei-Schichten-Architektur nach ANSI-SPARC beschreibt grob, wie Datenbanksysteme einzusetzen sind. Es beschreibt die Externe, Konzeptionelle und die interne Schicht. Die externe beschreibt die Benutzer und die abzubildende Welt, die konzeptionelle die Umsetzung in als Tabellen und die interne das zugrunde liegende Datenbanksystem. \\
Die skizzierte Architektur�bersicht zeigt eine konkrete Umsetzung eines MySQL Servers. Sie zeigt verschiedene Komponenten, die zusammen den SQL Server bilden. Im Bezug auf die Drei-Schichten-Architektur bildet die �bersicht nur einen Teil der inneren Schicht ab.
\end{itemize}

\end{document}