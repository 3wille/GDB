\documentclass[ngerman]{gdb-aufgabenblatt}


\renewcommand{\Aufgabenblatt}{1}
\renewcommand{\Ausgabedatum}{Mi. 15.10.2014}
\renewcommand{\Abgabedatum}{Do. 31.10.2014}
\renewcommand{\Gruppe}{M�ller, Meyer, Schulze}


\begin{document}

\section{Informationsmodellierung mit dem Entity-Relationship-
Modell}

%\begin{center}
%\begin{tikzpicture}
%\node[entity] (e1) {E1}
%\end{tikzpicture
%\end{center}


%\section{Beispiel f�r ER-Diagramm}

\begin{center}
\begin{tikzpicture}

\node[entity] (tier){Tier};
\node[attribut] (tier-a1) [above right=1cm and 5mm of tier.east] {\underline{Name}} edge (tier);
\node[attribut] (tier-a2) [right=1cm of tier.east] {\underline{\dashuline{Tierart Bezeichnung}}} edge (tier);

\node[entity] (ga)[left=1cm of tier] {Gattung};
\node[attribut] (ga-a1) [above left =5mm and 4mm of ga.north] {\underline{Gattungs Bezeichnung}} edge (ga);
\node[attribut] (ga-a2)  [left=5mm of ga.west] {Beschreibung} edge (ga);

\node[entity] (ti) [below left=3cm and 1cm of ga.south]  {Tierart};
\node[attribut] (ti-a1) [above left=1cm and 3cm of ti.north] {Lebenserwartung} edge (ti);
\node[attribut] (ti-a2) [left=1cm of ti.west] {\underline{Tierart Bezeichnung}} edge (ti);
\node[attribut] (ti-a1) [below left=1cm and 5mm of ti.west] {Beschreibung} edge (ti);

\node[relationship] (gati) [below left=1cm and 2cm of ga.south] {geh�rt an};
\path (gati) edge node[at end,anchor=north] {$n$} (ga);
\path (gati) edge node[at end,anchor=south] {$1$} (ti);

\node[relationship] (tierart) [below left=1cm and 3cm of tier.south ] {geh�rt an};
\path (tierart) edge node[at end,anchor=north] {$1$} (tier);
\path (tierart) edge node[at end,anchor=south] {$n$} (ti);

\node[relationship] (eltern) [above left =1cm and 1cm of tier.north ] {Eltern};
\path (eltern) edge node[at end,anchor=south] {$n$} (tier);
%\path (eltern) edge node[at end,anchor=west] {$n$} (tier);
\draw[double distance=2pt] (eltern) -- node[at end,anchor=east] {$n$} (tier);

\node[entity] (kaefig) [below right =3cm and 3cm of tier.south ] {K�fig};
\node[attribut] (kaefig-a1) [below right=1cm and 5mm of kaefig.south] {\underline{KID}} edge (kaefig);
\node[attribut] (kaefig-a2) [right=1cm of kaefig.east] {Fl�che} edge (kaefig);

\node[relationship] (lebt) [above left =1cm and 1cm of kaefig.north ] {lebt in};
\path (lebt) edge node[at end,anchor=north] {$1$} (tier);
\path (lebt) edge node[at end,anchor=south] {$n$} (kaefig);

\node[entity] (pfleger) [below left =5cm and 1cm of kaefig.south ] {Pfleger};
\node[attribut] (pfleger-a1) [below right=1cm and 5mm of pfleger.south] {\underline{SozNr}} edge (pfleger);
\node[attribut] (pfleger-a2) [right=1cm of pfleger.east] {Fl�che} edge (pfleger);
\node[attribut] (pfleger-a3) [left=1cm  of pfleger.west] {Vorname} edge (pfleger);
\node[attribut] (pfleger-a4) [below left=1cm and 1cm of pfleger.south] {Nachname} edge (pfleger);

\node[relationship] (lebt) [above right =1cm and 1cm of pfleger.north ] {k�mmert sich um};
\path (lebt) edge node[at end,anchor=south] {$n$} (pfleger);
\path (lebt) edge node[at end,anchor=north] {$1$} (kaefig);

\node[relationship] (lebt) [above right =1cm and 1cm of pfleger.north ] {k�mmert sich um};
\path (lebt) edge node[at end,anchor=south] {$n$} (pfleger);
\path (lebt) edge node[at end,anchor=north] {$1$} (kaefig);

\node[entity] (tat) [above left= 2cm and 5cm of pfleger.west ] {T�tigkeit};
\node[attribut] (tat-a1) [below left=1cm and 2cm of tat.south] {\underline{TID}} edge (tat);
\node[attribut] (tat-a2) [left=1cm of tat.west] {Dauer} edge (tat);
%\node[attribut] (tat-a3) [above left =1cm and 1cm of tat.north] {\dashuline{Pfleger}} edge (tat);Fl�che
%\node[attribut] (tat-a4) [right=5mm of tat.east] {\dashuline{Tier}} edge (tat);
%\node[attribut] (tat-a5) [below right=1cm and 5mm of tat.east] {Datum} edge (tat);

\node[entity] (pfvo) [above  left= 3cm and 15mm of pfleger.north ] {Pflegevorgang} edge [erbt] (tat);
\node[attribut] (pfvo-a1) [below left=1cm and 5mm of pfvo.south] {\dashuline{Pfleger}} edge (pfvo);
\node[attribut] (pfvo-a2) [above right=1cm and 5mm of pfvo.east] {\dashuline{Tier}} edge (pfvo);
\node[attribut] (pfvo-a3) [right=1cm of pfvo.east] {Datum} edge (pfvo);

\node[relationship] (hat getan) [above left =1cm and 1cm of pfleger.north ] {hat getan};
\path (hat getan) edge node[at end,anchor=south] {$n$} (pfleger);
\path (hat getan) edge node[at end,anchor=north] {$m$} (pfvo);

\node[relationship] (durchan) [above =1cm of pfvo.north ] {durchgef�hrt an};
\path (durchan) edge node[at end,anchor=south] {$n$} (pfvo);
\path (durchan) edge node[at end,anchor=north] {$1$} (tier);



%\node[multivalentattribut] (e1-a3)  [right=5mm of e1] {A3} edge (e1);
%
%\node[entity] (ti) {Tierart};
%\node[attribut] (ti-a1) [above left =5mm and 4mm of ti.north] {\underline{Bezeichnung}} edge (ti);
%\node[attribut] (ti-a2)  [above right=5mm and 4mm of ti.north] {Beschreibung} edge (t);

%\node[entity] (e3)      [below right=1cm and 1mm of e1.south] {E3} edge [erbt] (e1);
%\node[entity] (e4) [right =7cm of e3] {E4};

%\node[weakentity] (e5) [below =3cm of ti] {E5};
%\node[attribut] (e5-a1)  [right=5mm of e5] {\dashuline{A1}} edge (e5);
%
%\node[relationship] (r1) [right=2cm of e3] {R1};
%\path (r1) edge node[at end,anchor=north west] {$[0;2]$} (e3);
%\path (r1) edge node[at end,anchor=north east] {$[7;9]$} (e4);
%
%
%\node[weakrelationship] (r2) [below=1cm of e3] {R2};
%\path (r2) edge node[at end,anchor=north west] {$1$} (e3);
%\draw[double distance=2pt] (r2) -- node[at end,anchor=south east] {$8$} (e5);
%
%

\end{tikzpicture}
\end{center}

%
%
%
%
%
%\section{Beispiel f�r relationales Datenbankschema}
%
%\begin{RMSchma}
%Person(\soliduline{PID}, Name, Vorname)
%
%Haustier(\soliduline{HID}, Name, Rasse, \dashuline{Herrchen $\rightarrow$ Person.PID})
%\end{RMSchma}
%
%
%
%
%
%
%\section{Beispiel f�r Ausdruck der Relationenalgebra}
%
%\begin{align*}
% &\projektion{Rasse, Geschlecht}((Wolf\verbund{Wolf.WID=Haustier.HID} (\selektion{Name=\wert{Hasso}}Haustiere)) \natverbund Person)
%\\  &=\{ \wert{Steppenwolf}, \wert{m} \}
%\end{align*}
%
%
%
%
%\newpage
%\section{Beispiel f�r SQL-Anfrage}
%
%\begin{verbatim}
%SELECT 
%  h.Name,
%  h.Rasse
%FROM 
%  Haustier h,
%  Person p
%WHERE
%  h.Herrchen = p.PID AND
%  p.Vorname LIKE "P%"
%\end{verbatim}
%
%
%
%
%
%
%
%
%\section{Beispiel f�r Operatorbaum}
%
%\begin{tikzpicture}
%\node (Haustier) {Haustier};
%\node (Wolf) [left=25mm of Haustier] {Wolf};
%\node (join1) [above=20mm of $(Haustier)!.5!(Wolf)$] {$\verbund{Wolf.WID=Haustier.HID}$};
%\node (selektion1) [above=of join1] {$\selektion{Name=\wert{Hasso}}$};
%\node (projektion) [above=of selektion1] {$\projektion{Rasse}$};
%\node (final) [above=of projektion] {};
%
%\path (Haustier) edge node[smallr,near start,above right] {?? Tupel\\?? Attribute} (join1);
%\path (Wolf) edge node[smalll,near start,above left] {?? Tupel\\?? Attribute} (join1);
%\path (join1) edge node[smallr,near start,above left] {?? Tupel\\?? Attribute} (selektion1);
%\path (selektion1) edge node[smallr,midway,left] {$??\cdot\frac{??}{??}=??$ Tupel\\?? Attribute} (projektion);
%\path (projektion) edge node[smallr,midway,left] {$??$ Tupel\\1 Attribut} (final);
%\end{tikzpicture}
%
%
%
%
%
%
%
%\section{Beispiel f�r Tabelle mit Sperranforderungen}
%
%\begin{tabular}{|p{2cm}|p{2cm}|p{2cm}|p{2cm}|p{1cm}|p{1cm}|p{1cm}|p{3cm}|}
%\hline
%Zeitschritt & T\ts{1} & T\ts{2} & T\ts{3} & x & y & z & Bemerkung\\
%\hline
%\hline
%0 &  &  &  & NL & NL & NL & \\
%\hline
%1 & lock(x,X) &  &  & X\ts{1} & NL & NL & \\
%\hline
%2 & write(x) & lock(y,R) &  & X\ts{1} & R\ts{2} & NL & \\
%\hline
%3 &  &  &  &  &  &  & \\
%\hline
%4 &  &  &  &  &  &  & \\
%\hline
%5 &  &  &  &  &  &  & \\
%\hline
%\end{tabular}
%
%
%
%\section{*Thema*}
%
%*L�sung*






\end{document}