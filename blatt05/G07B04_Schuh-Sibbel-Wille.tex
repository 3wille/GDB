\documentclass[ngerman]{gdb-aufgabenblatt}

\usepackage{graphicx}
\usepackage{minted}
\renewcommand{\Aufgabenblatt}{5}
\renewcommand{\Ausgabedatum}{Mi. 10.12.2014}
\renewcommand{\Abgabedatum}{Fr. 09.01.2015}
\renewcommand{\Gruppe}{Schuh, Sibbel, Wille}

\usemintedstyle{tango}

\begin{document}

\section{Referentielle Aktionen}
\begin{enumerate}
\item
Ein Schema ist sicher, wenn referentielle Aktionen Reihenfolge-unabh\"angig sind, dass hei{\ss}t egal wie man die AKtionen ausf\"uhrt muss es das Selbe ergeben.
\item
\begin{itemize}
\item l{\"o}schen von leiht{\_}aus und in{\_}Bestand ohne weitere Konsequenzen
\item L{\"o}sche Videothek:\\
\begin{itemize}
\item erst Person
\begin{itemize}
\item Person: l{\"o}sche nur Videotheken, auf die keine Person eine Referenz hat
\item in{\_}Bestand: l{\"o}schen in in{\_}Bestand
\end{itemize}
\item erst in{\_}Bestand
\begin{itemize}
\item in{\_}Bestand: 
\item Person: 
\end{itemize}
\end{itemize}
$\rightarrow$ nicht richtungsabh\"angig
\item L{\"o}sche Film:
\begin{itemize}
\item 

\end{itemize}
\end{itemize}
\end{enumerate}

\section{{\"A}nderbarkeit von Sichten}
\begin{enumerate}
\item
\begin{enumerate}
\item
\begin{minted}{sql}
CREATE VIEW FerrariMechaniker
 AS SELECT Nachname, Vorname
  FROM Mechaniker, Rennwagen
   WHERE wartet = RNr
   AND Typ = 'Ferrari';
\end{minted}
\item
\begin{minted}{sql}
CREATE VIEW reicheMechaniker
 AS SELECT Vorname, Nachname 
  FROM Mechaniker
   WHERE Gehalt > 2 000 000;
\end{minted}
\item
\begin{minted}{sql}
CREATE VIEW alteRennserien
 AS SELECT Rennserie
  FROM Rennwagen
   WHERE Jahr > 1950;
\end{minted}
\item
\begin{minted}{sql}
CREATE VIEW FerrariWagen
 AS SELECT * FROM Rennwagen
  WHERE Rennstall = 'Ferrari';
\end{minted}
\end{enumerate}
\item
\begin{enumerate}
\item
Die {\"a}nderung ist zul{\"a}ssig und wird in Formel1 Wagen und Auto-Union-Rennwagen komplett sichtbar sein
\item
Die {\"a}nderung funktioniert nicht, da alle Eintr{\"a}ge in dieser View mit der {\"a}nderung der Jahres zu 2014 aus dem gesetzten Zeitfenster (1961-1963) fallen w{\"u}rden.
\item
Auch hier w{\"u}rde das Auswahlkriterium (Rennstall von Ferrari zu Lotus) ge{\"a}ndert werden, die {\"a}nderung ist damit nicht m{\"o}glich.
\item
Es kann nicht eingef{\"u}gt werden,da bei Auto-Union-Rennwagen die Wagen alle aus der Formel 1 stammen m{\"u}ssen, der einzuf{\"u}gende aber in der Rennserie AVUS f{\"a}hrt. 
\end{enumerate}
\end{enumerate}
\section{Serialisierbarkeit, Anomalien}
\begin{enumerate}
\item 
\begin{enumerate}
\item[$S_1$:]A=320 B=10
\item[$S_2$:]A=315 B=220
\item[$S_3$:]A=520 B=110
\item[$S_4$:]A=215 B=10
\item[$S_5$:]A=110 B=10
\item[$S_6$:]A=520 B=110
\end{enumerate}
\item
\begin{enumerate}

\end{enumerate}
\end{enumerate}

\section{2PL-Synchronisation mit R/X-Sperren}
% S1 = w1(x) r2(z) r1(x) r3(y) w2(y) w1(y) c1 r2(z) c2 w3(y) c3
 \begin{tabular}{|c|c|c|c||c|}
 \hline
    & $T_1$ & $T_2$ & $T_2$ & Bemerkung \\ \hline 
	1 & lock(x,X) &  &  &   \\
    2 & write(x) & lock(z,R) &  &   \\
    3 & read(x) & read(z) & lock(y,R) & $T_2$ wartet auf Freigabe von y \\
    4 &  &  & read(y) & $T_1$ wartet auf Freigabe von y  \\
    5 &  &  &  &   \\
    6 &  &  &  &   \\
    7 &  &  &  &   \\
    8 &  &  &  &   \\
    9 &  &  &  &   \\
    10 &  &  &  &   \\
    11 &  &  &  &   \\
    12 &  &  &  &   \\
    13 &  &  &  &   \\
    14 &  &  &  &   \\
    15 &  &  &  &   \\
    16 &  &  &  &   \\
    17 &  &  &  &   \\
    18 &  &  &  &   \\
    19 &  &  &  &   \\
    20 &  &  &  &   \\
    21 &  &  &  &   \\
    22 &  &  &  &   \\
    23 &  &  &  &   \\ \hline
 \end{tabular}
\end{document}