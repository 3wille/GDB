\documentclass[ngerman]{gdb-aufgabenblatt}

\usepackage{graphicx}
\usepackage{minted}
\renewcommand{\Aufgabenblatt}{5}
\renewcommand{\Ausgabedatum}{Mi. 10.12.2014}
\renewcommand{\Abgabedatum}{Fr. 09.01.2015}
\renewcommand{\Gruppe}{Schuh, Sibbel, Wille}

\usemintedstyle{tango}

\begin{document}
\section{ }
\section{ }
\section{Normalformlehre}
\begin{enumerate}
\item 
M{\"o}gliche Schl{\"u}sselkandidaten sind jeweils C und D, da von beiden auf alle anderen Attribute geschlossen werden kann und 1 Attribut als Schl{\"u}ssel immer minimal ist.
\item
Alle anderen.
\item
Da die Attribute atomar sind, ist die Relation auf jeden Fall in der ersten Normalform.\\
Sei D als Prim{\"a}rschl{\"u}ssel gew{\"a}hlt, ist die Relation in der zweiten Normalform, da der Prim{\"a}rschl{\"u}ssel aus nur einem Attribut besteht und somit kein Attribut von nur einem Teil des Schl{\"u}ssel abh{\"a}ngig sein kann.\\
Die Relation ist nicht in dritter Normalform, da E nur von C und nicht dem Schl{\"u}ssel abh{\"a}ngig ist.
\end{enumerate}

\end{document}